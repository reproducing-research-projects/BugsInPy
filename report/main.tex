\documentclass[conference]{IEEEtran}
%\IEEEoverridecommandlockouts
% The preceding line is only needed to identify funding in the first footnote. If that is unneeded, please comment it out.

\usepackage[backend=biber,style=ieee]{biblatex}

\addbibresource{main.bib}
\addbibresource{IEEEabrv.bib}

\input{pandoc_header.tex}

\begin{document}

\title{Reproducing BugsInPy}

\author{\IEEEauthorblockN{Faustino Aguilar}
\IEEEauthorblockA{\textit{dept. name of organization (of Aff.)} \\
University of Panama \\
Panama City, Panama \\
\textit{email address or ORCID}}
\and
\IEEEauthorblockN{Samuel Grayson}
\IEEEauthorblockA{Dept. of Computer Science \\
University of Illinois at Urbana Champaign \\
Urbana, IL \\
https://orcid.org/0000-0001-5411-356X}
\and
\IEEEauthorblockN{Darko Marinov}
\IEEEauthorblockA{Dept. of Computer Science \\
University of Illinois at Urbana Champaign \\
Urbana, IL \\
https://orcid.org/0000-0001-5023-3492}
}

\maketitle

\begin{abstract}
  We present our experience on replicating a bug dataset for the Python programming language.
  We assess the reproducibility of the original dataset less than three years after its original publication.
  The bug dataset provides some information about the software environment, but this information can be incomplete or it can decay into something uninstallable.
  We rectify as many of these problems as we can and redesign the original dataset to be more easily reusable and reproducible by future authors.
  Based on our experience, we offer suggestions to Python artifact authors to improve their reproducibility. 
\end{abstract}

\begin{IEEEkeywords}
reproducibility, bug database, python
\end{IEEEkeywords}

% http://www.ieee-scam.org/2023/#cfpengtrack

\input{meat.tex}

\section*{References}

\printbibliography

\appendix[Code, Data, and Reproducing]

A snapshot of the latest state of this code can be found at: \url{... (Zenodo DOI)}.

A rolling release of the code can be found at: \url{... (GitHub)}.

In the rolling release or snapshot:
\begin{itemize}
\item \texttt{data} holds a machine-readable view of the data, split across several files.
\item \texttt{spack/spack.lock} contains the Spack environment in which this experiment was run.
\end{itemize}

To reproduce this paper, run:

\begin{verbatim*}
# command
output
\end{verbatim*}

After which, the results will be here:

\begin{itemize}
\item \texttt{reports/main.pdf} This is the actual paper.
\item \texttt{raw\_data} This is the raw data.
\end{itemize}

\end{document}
