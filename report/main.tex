\documentclass[conference]{IEEEtran}
%\IEEEoverridecommandlockouts
% The preceding line is only needed to identify funding in the first footnote. If that is unneeded, please comment it out.

\usepackage[backend=biber,style=ieee]{biblatex}

\addbibresource{main.bib}
\addbibresource{IEEEabrv.bib}

\input{pandoc_header.tex}

\begin{document}

\title{Reproducing BugsInPy}

\author{\IEEEauthorblockN{1\textsuperscript{st} Given Name Surname}
\IEEEauthorblockA{\textit{dept. name of organization (of Aff.)} \\
\textit{name of organization (of Aff.)}\\
City, Country \\
email address or ORCID}
\and
\IEEEauthorblockN{2\textsuperscript{nd} Given Name Surname}
\IEEEauthorblockA{\textit{dept. name of organization (of Aff.)} \\
\textit{name of organization (of Aff.)}\\
City, Country \\
email address or ORCID}
\and
\IEEEauthorblockN{3\textsuperscript{rd} Given Name Surname}
\IEEEauthorblockA{\textit{dept. name of organization (of Aff.)} \\
\textit{name of organization (of Aff.)}\\
City, Country \\
email address or ORCID}
\and
\IEEEauthorblockN{4\textsuperscript{th} Given Name Surname}
\IEEEauthorblockA{\textit{dept. name of organization (of Aff.)} \\
\textit{name of organization (of Aff.)}\\
City, Country \\
email address or ORCID}
\and
\IEEEauthorblockN{5\textsuperscript{th} Given Name Surname}
\IEEEauthorblockA{\textit{dept. name of organization (of Aff.)} \\
\textit{name of organization (of Aff.)}\\
City, Country \\
email address or ORCID}
\and
\IEEEauthorblockN{6\textsuperscript{th} Given Name Surname}
\IEEEauthorblockA{\textit{dept. name of organization (of Aff.)} \\
\textit{name of organization (of Aff.)}\\
City, Country \\
email address or ORCID}
}

\maketitle

\begin{abstract}
  We present our experience on replicating a bug dataset for the Python programming language.
  The bug dataset provides some information about the software environment, but this environment decays quickly into something uninstallable.
  We assess the reproducibility over time and improve the reproducibility of the dataset.
\end{abstract}

\begin{IEEEkeywords}
reproducibility, bug database, python
\end{IEEEkeywords}

% http://www.ieee-scam.org/2023/#cfpengtrack

\input{meat.tex}

\section*{References}

\printbibliography

\appendix[Code, Data, and Reproducing]

A snapshot of the latest state of this code can be found at: \url{... (Zenodo DOI)}.

A rolling release of the code can be found at: \url{... (GitHub)}.

In the rolling release or snapshot:
\begin{itemize}
\item \texttt{data} holds a machine-readable view of the data, split across several files.
\item \texttt{spack/spack.lock} contains the Spack environment in which this experiment was run.
\end{itemize}

To reproduce this paper, run:

\begin{verbatim*}
# command
output
\end{verbatim*}

After which, the results will be here:

\begin{itemize}
\item \texttt{reports/main.pdf} This is the actual paper.
\item \texttt{raw\_data} This is the raw data.
\end{itemize}

\end{document}
